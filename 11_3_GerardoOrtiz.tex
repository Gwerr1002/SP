\documentclass[12pt]{article}
\usepackage[utf8]{inputenc}
\usepackage{amsmath}
\usepackage{amsfonts}
\usepackage{amssymb}
\usepackage{vmargin} %configuraciòn de la hoja
\setpapersize{A4} %tamaño del papel
\setmargins{2cm} %margen izquierdo
{1cm}%margen superior
{17cm} %ancho del área de impresión
{25cm} %longitud del área de impresión
{0pt}%espaciado para el encabezado
{10mm} %espacio entre el encabezado y el texto
{0pt} %espacio entre el pie de página
{5mm} %espacio entre el texto y el pie de página
\usepackage{amsmath}
\usepackage{amsfonts}
\usepackage{amssymb}
\usepackage{graphicx}
\usepackage{lipsum}
\usepackage{natbib}
\usepackage{hyperref}

\usepackage{xcolor}

\definecolor{v}{rgb}{0,255,0}
\definecolor{verde}{HTML}{64B235}
\definecolor{lima}{HTML}{AEE51F}
\definecolor{sapo}{HTML}{2E7310}
\definecolor{limon}{HTML}{41AA13}

\definecolor{a}{rgb}{0,0,255}
\definecolor{azul}{HTML}{188FE4}
\definecolor{cian}{HTML}{14D4B2}
\definecolor{aqua}{HTML}{0FD283}
\definecolor{rey}{HTML}{1755CC}

\definecolor{r}{rgb}{255,0,0}
\definecolor{rojo}{HTML}{F43434}
\definecolor{carmesi}{HTML}{F9401F}
\definecolor{naranja}{HTML}{F67A13}
\definecolor{melon}{HTML}{F6B780}
\definecolor{mandarina}{HTML}{F39716}

\definecolor{amarillo}{HTML}{FFDD00}
\definecolor{huevo}{HTML}{EDD220}
\definecolor{mostaza}{HTML}{EDB40B}
\definecolor{champagne}{HTML}{F6ECA0}
\definecolor{beige}{HTML}{E2D99C}

\definecolor{cafe}{HTML}{AB7A42}
\definecolor{cocoa}{HTML}{895B26}
\definecolor{chocolate}{HTML}{A36B2A}

\definecolor{morado}{HTML}{7030E0}
\definecolor{lila}{HTML}{B38AFA}

\definecolor{magenta}{HTML}{D827F7}
\definecolor{rosa}{HTML}{F180DE}
\definecolor{rosamex}{HTML}{EF29A3}
\definecolor{pig}{HTML}{F374C2}
\definecolor{escarlata}{HTML}{DC0743}

\definecolor{backcolour}{HTML}{4C5150}
\definecolor{codegreen}{rgb}{0,0.6,0}
\definecolor{codegray}{HTML}{BEC6C6}

\definecolor{gris1}{HTML}{5E6161}

\title{\textbf{SP 22O\\Entrega semanal 2}}
\author{Gerardo Ortíz Montúfar}
\date{Octubre 27, 2022}

\begin{document}
\maketitle

\textbf{Proyecto: Medición de la cantidad de grasa fetal a través
de ultrasonido y Deep Learning}

\section{Analizando el estado de la cuestión}

\begin{enumerate}
	\item \textbf{En todo problema de investigación existe un conocimiento previo (antecedentes) a partir del cual se
formulan interrogantes. Una interrogante es significativa con relación al conocimiento de cierto objeto
de estudio siempre que identifique: dificultades, contradicciones, debilidades o vacíos. Analiza el
contenido de alguna(s) de las referencias que forme(n) parte del estado del arte de tu proyecto (lo que
ya se ha hecho alrededor del tema específico u objeto de estudio) y redacta una pregunta de
investigación resultado de ese análisis. Menciona también si se relaciona con dificultades,
contradicciones, debilidades y/o vacíos alrededor de ese conocimiento previo plasmado en la referencia
bajo análisis (fundamenta tu respuesta).}
	
	\vspace{5mm}
	En torno a este proyecto se han realizado diversas
	técnicas para la detección temprana del IUGR (Intrauterine
	Groth Restriction) que incluyen cardiotocografía, ecografía
	con Doppler, medición del peso fetal mediante medidas
	biométricas. Sin embargo, estos tienen cierta tasa de 
	medición.
	Los métodos de palpación abdominal y medición de la altura 
	del fondo  tienen bajas tasas de detección del IUGR. 
	
	Las 
	mediciones a través de Doppler para evaluación del bienestar
	fetal y detección temprana de IUGR como el de la 
	arteria uterina, umbilical, cerebral, índice 
	cerebro-placentario, del conducto venoso y del itsmo aórtico.
	Tiene poca sensibilidad y especificidad para predecir
	resultados adversos del SGA (Small for Gestational Age)\citep{sharma2016intrauterine}.
	Esto nos da la necesidad de buscar más métodos que nos generen
	una predicción más certera de la restricción de crecimiento
	para evitarla.
	
	Sumado a esto los exámenes clínicos de recién nacidos con
	IUGR tienen variadas características de desnutrición de las
	cuales podemos en listar las siguientes
	\citep{sharma2016intrauterine}:
	\begin{itemize}
		\item Cabeza grande en comparación con el cuerpo..
		\item Grasa bucal ausente.
		\item Disminución de la masa muscular esquelética y grasa
		subcutánea.
		\item Piel floja.
		\item Uñas largas.
		\item Bebé ansioso e hiperalerta.
		\item Genitales femeninos de apariencia menos madura.
		debido a la falta de grasa.
	\end{itemize}
	
	Tomando esto en cuenta se ha propuesto medir la cantidad de
	grasa como un posible indicador para predecir el IUGR y 
	evitarlo De este modo la pregunta de investigación sería 
	¿Por medio de la medición	de la grasa fetal es posible
	predecir la	restricción de crecimiento?
	
	La primer complicación con la que nos encontramos es cómo
	medir la grasa fetal por lo que hay que desarrollar métodos
	con los que podamos cuantificarla para responder la pregunta
	planteada anteriormente. Tampoco se tiene garantía de que 
	se pueda cuantificar por medio de Deep Learning en ultra 
	sonido.
	
	\item \textbf{Elige uno de los documentos que has considerado como referencia bibliográfica para plantear tu
proyecto (súbelo a tu carpeta) y menciona en cuál de las siguientes categorías se encuentra (fundamenta
tu respuesta)}
	\begin{enumerate}
		\item \underline{Resume el conocimiento previo sobre el tema de tu proyecto y te ayudará para el} 
		
		\underline{planteamiento,
formulación y afinamiento del problema que abordarás.}
		\item Se acerca en gran medida al planteamiento del problema específico que tienes a la fecha.
	\end{enumerate}
	
	Título: Intrauterine Growth Restriction: Antenatal and Postnatal
Aspects

	Elegí el inciso a) ya que el artículo describe qué es la 
	restricción del crecimiento intrauterino, sus posibles causas,
	de que forma se puede predecir, prevenir y diagnosticar. Así
	como las consecuencias que esta patología puede traerle en 
	su etapa de desarrollo posnatal y una breve introducción a las
	teorías más aceptadas que respaldan la información contenida
	en el artículo.
	
	\item \textbf{Propón dos frases que utilizarás en un buscador para actualizar tu revisión sobre el estado de la cuestión
de tu proyecto (fundamenta tu respuesta).}
	\begin{itemize}
		\item Arquitecturas de Deep Learning: Para catalogar
		cuales pueden servir para este propósito, me han sugerido 
		Resnet50, VGG19, Autoencoder. La pregunta aquí es 
		¿Qué arquitectura es más optima o me ayudará a lograr la
		meta?
		
		\item Medición de grasa fetal y posparto en IUGR:
		Para extender la información que poseo sobre la masa
		grasa en bebés recién nacidos y por si hay alguna 
		actualización de algún método que aporte a este proyecto
	\end{itemize}
\end{enumerate}

\section{Analizando “El título”, “El tema” y “Las delimitaciones temporales y espaciales”}

\begin{enumerate}
	\item \textbf{Elige un artículo o tesis relacionados con el tema de tu proyecto (o de tu interés en caso de que todavía
no tengas elegido tu proyecto) y agrega la referencia, la liga o el propio artículo a la carpeta con tu
nombre. A partir de la lectura del título, el resumen y/o la introducción contesta a las siguientes
preguntas:}

Título: Intrauterine Growth Restriction: Antenatal and Postnatal
Aspects

	\begin{itemize}
		\item ¿El título tiene las características que se mencionan en el libro de Salmerón y Suárez?
Fundamenta tu respuesta.
		
				Identifico al problema el cuál es la restricción
				intrauterina del crecimiento y el tema que son 
				los aspectos antenatal y posnatal. En el título
				se observa temporalidad al subrayar que se
				detallan aspectos antes y después del parto.
				Por otro lado, no observo espacialidad ya que no
				se especifica un lugar. Aunque el artículo es Indi
		
		\item ¿Cuál dirías que fue la cuestión particular a estudiar? (es decir, los aspectos de un fenómeno o
proceso cuya investigación mereció la pena comprometerse) fundamenta tu respuesta.

		Los aspectos de un fenómeno ya que se describe
		detalladamente una patología incluyendo causas,
		consecuencias, prevención, predicción, tipos (variantes
		del fenómeno), patrones (signos)

		\item ¿Identificas alguna mención de algo que se podría considerar como límites temporales o
espaciales? Fundamenta tu respuesta
		
		Establecen las descripciones antes y después del embarazo
		por lo que es un límite temporal. Espaciales no están
		delimitados pero dentro de los institutos implicados 
		muchos son de la India
		
		\item ¿Se enuncian actores, instituciones o lugares?
		
		Sí, los principales actores o autores se mencionan en el
		artículo así como las instituciones implicadas. Dentro del
		documento
		
		\item Con base en el análisis del resumen/introducción ¿El autor cumple con lo que en el título
promete haber estudiado? ¿Delimita claramente el problema a resolver? Fundamenta tu respuesta
		
		Abarca los puntos esperados, aunque no los delimita del
		todo y al leer el atículo hay más información de la
		esperada. Lo cuál no es un punto en contra, si no
		enriquece el texto, pero convendría mencionar algunos
		aspectos a tratar en la introducción de otra forma 
		alguien podría dejarlo de lado al leer sólo la
		introducción.

	\end{itemize}
\end{enumerate}

\setcitestyle{author, \open{((},\close{))}}%formato autor ,(año)
\bibliographystyle{apalike} %referencias APA sin ningún paquete
\bibliography{ref}

\end{document}
